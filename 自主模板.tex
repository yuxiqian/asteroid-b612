\documentclass{mcmthesis}
\mcmsetup{CTeX = true,   % 使用 CTeX 套装时,设置为 true
        tcn = 0038
        , problem = C,
        sheet = true, titleinsheet = true, keywordsinsheet = true,
        titlepage = true, abstract = true}
\usepackage{palatino}
\usepackage{lipsum}
\title{The \LaTeX{} Template for MCM Version \MCMversion}
\author{}
\date{\today}
\begin{document}
\begin{abstract}
fhakfhw
\begin{keywords}
keyword1; keyword2
\end{keywords}
\end{abstract}
\maketitle
\tableofcontents

\newpage
\section{Introduction}

\subsection{Problem Statement}
In the modern life, people would like to live in a healthy lifestyle. Then it is unavoidable to buy fresh and cheap fruits. Fruit prices fluctuate significantly from week to week. Consumers sometimes may be sensitive to these changes while sometimes not, leading to some fruits selling well while some do not. Besides, when seasons change, the popularity of some fruits also decline. Other factors can also influence the selling of fruits. When customers buy less fruits, those redundant fruits may soon become rotted and cause merchants’ loss. To avoid too many orders, fruit merchants have to predict how much fruits to be sold during the following days and try to make these predictions as close to the reality as possible. Moreover, when the selling of fruits does not live up to their expectations, they may use various strategies to promote, such as discount, combination, VIP benefits and etc.
\subsection{Related Work}

\section{Assumptions and Notations}

\subsection{Assumptions}
We make the following basic assumptions in order to simplify the problem. Each
of our assumptions is justified and is consistent with the basic fact.
\begin{itemize}
\item \textbf{}
\item \textbf{}
\item \textbf{}
\item \textbf{}
\end{itemize}
\subsection{Notations}
\begin{table}
    \centering
    \caption{Notations} \label{table-notations}
    \begin{tabular}{ccc}
    \toprule
        Symbol & Definition & Notes\\
    \midrule
        h & hh & hhh\\ 
    \bottomrule
    \end{tabular}
\end{table}
\section{Model Construction}

\subsection{Model 1}
\subsection{Model 2}
\begin{figure}[h]
\small
\centering
%\includegraphics[width=12cm]{mcmthesis-aaa.eps}
\caption{aa} \label{fig:aa}
\end{figure}

\begin{equation}
a^2 \label{aa}
\end{equation}

\[
  \begin{pmatrix}{*{20}c}
  {a_{11} } & {a_{12} } & {a_{13} }  \\
  {a_{21} } & {a_{22} } & {a_{23} }  \\
  {a_{31} } & {a_{32} } & {a_{33} }  \\
  \end{pmatrix}
  = \frac{{Opposite}}{{Hypotenuse}}\cos ^{ - 1} \theta \arcsin \theta
\]


\[
  p_{j}=\begin{cases} 0,&\text{if $j$ is odd}\\
  r!\,(-1)^{j/2},&\text{if $j$ is even}
  \end{cases}
\]

\[
  \arcsin \theta  =
  \mathop{{\int\!\!\!\!\!\int\!\!\!\!\!\int}\mkern-31.2mu
  \bigodot}\limits_\varphi
  {\mathop {\lim }\limits_{x \to \infty } \frac{{n!}}{{r!\left( {n - r}
  \right)!}}} \eqno (1)
\]

\section{Data Processing}

\subsection{Data Analysis}

All fruit selling data are limited as two comma separated value files,
hence all the following analyses are based on that two files.

The first file contains all purchasing records between the end of 2017 and the beginning of 2018, whereas 81445 records in approximately 3 months. Each record has a unique number as a ticket bill, carrying its purchasing time, merchant ID, paying method and special bonus.

The second file contains all 182868 goods that matches the ticket bills above, carrying the good's primary kind, selling amount and the discount information.

First thing needs to be done would be making connections between the bills and the goods. Since the unique ID, we can find all goods purchased in a single bill. The
final data would be fitted in the following structure:

\begin{table}
    \centering
    \caption{The data example} \label{table-data_example}
    \begin{tabular}{|c|p{20em}|}
    \hline
        ID & 28221526017700008\\
    \hline
        Time & 2018-02-28 22:15:26\\
    \hline
        PayMethod & Cash\\
    \hline
        Price & [0.982] Unit [Hainan Cherry Tomatoes] as [Season Fruit], Origin price [12.90] Discount Price [12.90]\\
    \hline
        isVip & False\\
    \hline
        SolderId & Merchant 22\\

    \hline
    \end{tabular}
\end{table}

\subsection{Error Data Fix}

Some of the purchasing method were mistagged or missed entirely. Those data are either moved in the correct type or in a special "unmarked" type.

Some fruit type are not marked correctly as so, and the similar solution is made, too.

Because of the binary stored float numbers can't be exact, all current numbers (the origin price, discount price, special bonus) are all truncated to 2 digits after the decimal dots.
\section{Model Extension and Simulation Analysis}
\subsection{Problem 1}

\subsubsection{Sensitivity Analysis}

\section{Strengths and weaknesses}

\section{Conclusion}

\subsection{Strengths}
\begin{itemize}
\item \textbf{Applies widely}\\
This  system can be used for many types of airplanes, and it also
solves the interference during  the procedure of the boarding
airplane,as described above we can get to the  optimization
boarding time.We also know that all the service is automate.
\item \textbf{Improve the quality of the airport service}\\
Balancing the cost of the cost and the benefit, it will bring in
more convenient  for airport and passengers.It also saves many
human resources for the airline. \item \textbf{}
\end{itemize}

\subsection{Weaknesses}
\begin{itemize}
\item




sfhawfh
\end{itemize}

\begin{thebibliography}{99}
\bibitem{1} D.~E. KNUTH   The \TeX{}book  the American
Mathematical Society and Addison-Wesley
Publishing Company , 1984-1986.
\bibitem{2}Lamport, Leslie,  \LaTeX{}: `` A Document Preparation System '',
Addison-Wesley Publishing Company, 1986.
\bibitem{3}\url{http://www.latexstudio.net/}
\bibitem{4}\url{http://www.chinatex.org/}
\end{thebibliography}

\begin{appendices}

\section{First appendix}

\lipsum[13]

Here are simulation programmes we used in our model as follow.\\

\textbf{\textcolor[rgb]{0.98,0.00,0.00}{Input matlab source:}}
\lstinputlisting[language=Matlab]{./code/mcmthesis-matlab1.m}

\section{Second appendix}

some more text \textcolor[rgb]{0.98,0.00,0.00}{\textbf{Input C++ source:}}
\lstinputlisting[language=C++]{./code/mcmthesis-sudoku.cpp}


\end{appendices}




\end{document}
